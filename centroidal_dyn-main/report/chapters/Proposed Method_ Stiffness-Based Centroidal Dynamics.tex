\documentclass[main.tex]{subfiles}

\begin{document}

\begin{sloppypar}
\section{Proposed Method}
\label{sec:proposedmethods}
\subsection{Stiffness-Based Centroidal Dynamics}

We begin from the standard \emph{centroidal dynamics} equations, which relate the motion of the robot’s center of mass (CoM) to the total external wrench (force and moment) acting on the system:
\begin{subequations}\label{eq:centroidal‐dynamics}
\begin{align}
m\,\ddot{\mathbf{p}} &= \mathbf{f} \;-\; m\,\mathbf{g}, \label{eq:centroidal‐dynamics:trans}\\
\dot{\mathbf{L}}     &= \boldsymbol{\eta}.      \label{eq:centroidal‐dynamics:rot}
\end{align}
\end{subequations}
Here:
\begin{itemize}
  \item $\mathbf{p}\in\mathbb{R}^3$ is the position of the CoM, and $\ddot{\mathbf{p}}$ its acceleration.
  \item $\mathbf{L}\in\mathbb{R}^3$ is the total angular momentum about the CoM.
  \item $\mathbf{f}\in\mathbb{R}^3$ and $\boldsymbol{\eta}\in\mathbb{R}^3$ are, respectively, the translational and rotational components of the total external wrench.
  \item $m>0$ is the total mass, and $\mathbf{g}\in\mathbb{R}^3$ is the gravity vector (e.g.\ $\mathbf{g}=[0,0,-9.81]^\top$).
\end{itemize}

We assume all external wrenches are contact wrenches at $n_e$ end‐effectors (``ends'') of the robot.  Denote by $\mathbf{p}_l\in\mathbb{R}^3$ the world‐frame position of the $l$-th end, and let
\[
\mathbf{f}_l\in\mathbb{R}^3,\quad
\boldsymbol{\eta}_l\in\mathbb{R}^3,\qquad l=1,\dots,n_e
\]
be the translational and rotational components of the contact wrench at that end.  Then the total wrench is
\begin{equation}\label{eq:sum‐wrench}
\mathbf{f}    \;=\;\sum_{l=1}^{n_e} \mathbf{f}_l,
\qquad
\boldsymbol{\eta} \;=\;\sum_{l=1}^{n_e}\Bigl[(\mathbf{p}_l - \mathbf{p})\times \mathbf{f}_l \;+\;\boldsymbol{\eta}_l\Bigr].
\end{equation}
The cross‐product in \eqref{eq:sum‐wrench} makes the system \emph{bilinear} in $(\mathbf{p},\mathbf{f}_l)$, coupling CoM motion with contact forces.

\medskip
\paragraph{Spring-like parametrization of contact wrenches.}
Rather than holding $\mathbf{f}_l,\boldsymbol{\eta}_l$ constant, we introduce a \emph{stiffness} parameter $\lambda_l\ge0$ and a \emph{CMP‐offset} vector $\mathbf{r}_l\in\mathbb{R}^3$ for each end, plus a pure‐moment direction $\mathbf{\hat{\boldsymbol{\eta}}}_l\in\mathbb{R}^3$.  Inspired by a spring model, we set:
\begin{equation}\label{eq:stiff‐param}
\boxed{
\displaystyle
\mathbf{f}_l = m\,\lambda_l^2\Bigl(\mathbf{p} \;-\;(\mathbf{p}_l + \mathbf{r}_l)\Bigr),
\qquad
\boldsymbol{\eta}_l = m\,\lambda_l^2\,\mathbf{\hat{\boldsymbol{\eta}}}_l.
}
\end{equation}
Intuitively:
\begin{itemize}
  \item $\lambda_l^2$ scales like a contact stiffness: larger $\lambda_l$ → stronger repulsive force.
  \item $\mathbf{p}_l + \mathbf{r}_l$ is the \emph{virtual pivot} (similar to a Centroidal Moment Pivot, CMP).  The force pulls the CoM toward that point.
  \item $\mathbf{\hat{\boldsymbol{\eta}}}_l$ encodes any pure moment about the CoM, scaled consistently by $\lambda_l^2$.
\end{itemize}

\subsubsection*{Derivation of the stiffness‐based model}
By substituting \eqref{eq:sum‐wrench} and \eqref{eq:stiff‐param} into \eqref{eq:centroidal‐dynamics:trans} and neglecting $O(\epsilon^2)$ terms one obtains
\begin{align}
m\,\ddot{\mathbf{p}}
&= \sum_{l=1}^{n_e} m\,\lambda_l^2\bigl(\mathbf{p} - (\mathbf{p}_l + \mathbf{r}_l)\bigr) \;-\; m\,\mathbf{g}
\notag\\
&\approx \sum_{l=1}^{n_e} m\,\lambda_l^2\bigl(\mathbf{p} - (\mathbf{p}_l + \mathbf{r}_l)\bigr) \;-\; m\,\mathbf{g} \;+\; m\,\epsilon^2 \mathbf{p}
\notag\\
&= m\Bigl(\sum_{l}\lambda_l^2 + \epsilon^2\Bigr)\mathbf{p} \;-\; m\Bigl(\sum_{l}\lambda_l^2(\mathbf{p}_l + \mathbf{r}_l) + \mathbf{g}\Bigr)
\notag\\
&= m\,\bar\lambda^2\bigl(\mathbf{p} - \mathbf{\bar{p}} - \mathbf{\bar{r}}\bigr),
\label{eq:deriv‐trans}
\end{align}
where
\[
\bar\lambda^2 = \sum_{l}\lambda_l^2 + \epsilon^2,
\quad
\mathbf{\bar{p}} = \frac{\sum_{l}\lambda_l^2\,\mathbf{p}_l + \mathbf{g}}{\bar\lambda^2},
\quad
\mathbf{\bar{r}} = \frac{\sum_{l}\lambda_l^2\,\mathbf{r}_l}{\bar\lambda^2}.
\]

Similarly, substituting into \eqref{eq:centroidal‐dynamics:rot} gives
\begin{align}
\dot{\mathbf{L}}
&= \sum_{l=1}^{n_e}\bigl[(\mathbf{p}_l - \mathbf{p})\times m\,\lambda_l^2(\mathbf{p} - \mathbf{p}_l - \mathbf{r}_l) + m\,\lambda_l^2\,\mathbf{\hat{\boldsymbol{\eta}}}_l\Bigr]
\notag\\
&= \sum_{l}\bigl[(\mathbf{p} - \mathbf{p}_l)\times m\,\lambda_l^2\,\mathbf{r}_l\bigr]
   \;+\;\sum_{l}m\,\lambda_l^2\,\mathbf{\hat{\boldsymbol{\eta}}}_l
\notag\\
&= \mathbf{p} \times m\,\bar\lambda^2\,\mathbf{\bar{r}}
   \;+\;\sum_{l}m\,\lambda_l^2\bigl(\mathbf{\hat{\boldsymbol{\eta}}}_l - \mathbf{p}_l\times \mathbf{r}_l\bigr)
\notag\\
&\approx \bigl(m\,\ddot{\mathbf{p}} + m\,\bar\lambda^2(\mathbf{\bar{p}}+\mathbf{\bar{r}})\bigr)\times \mathbf{\bar{r}}
   \;+\;\sum_{l}m\,\lambda_l^2\bigl(\mathbf{\hat{\boldsymbol{\eta}}}_l - \mathbf{p}_l\times \mathbf{r}_l\bigr)
\notag\\
&= m\bigl(\ddot{\mathbf{p}}\times \mathbf{\bar{r}} + \mathbf{\bar{\boldsymbol{\eta}}}\bigr),
\label{eq:deriv‐rot}
\end{align}
where
\[
\mathbf{\bar{\boldsymbol{\eta}}}
= \bar\lambda^2(\mathbf{\bar{p}}\times \mathbf{\bar{r}})
  + \sum_{l}\lambda_l^2\bigl(\mathbf{\hat{\boldsymbol{\eta}}}_l - \mathbf{p}_l\times \mathbf{r}_l\bigr).
\]

\medskip
\paragraph{Closed‐form centroidal dynamics.}
Combining \eqref{eq:deriv‐trans} and \eqref{eq:deriv‐rot} yields the stiffness‐based centroidal equations:
\begin{subequations}\label{eq:stiff‐centroidal}
\begin{align}
\ddot{\mathbf{p}} &= \bar\lambda^2\bigl(\mathbf{p} - (\mathbf{\bar{p}} + \mathbf{\bar{r}})\bigr),
\label{eq:stiff‐centroidal:trans}\\
\dot{\mathbf{L}} &= m\bigl(\ddot{\mathbf{p}} \times \mathbf{\bar{r}} + \mathbf{\bar{\boldsymbol{\eta}}}\bigr).
\label{eq:stiff‐centroidal:rot}
\end{align}
\end{subequations}
 
\noindent
The aggregated parameters $\bar\lambda,\mathbf{\bar{p}},\mathbf{\bar{r}},\mathbf{\bar{\boldsymbol{\eta}}}$ recover the same expressions as before. 

\paragraph{Discussion and special cases.}
\begin{remark}[Exactness vs.\ flight phase]
If one ignores flight (i.e.\ always in contact, $\sum_l\lambda_l^2>0$), one may set $\epsilon=0$ and \eqref{eq:stiff‐centroidal} hold exactly.  Otherwise $\epsilon>0$ guarantees a well‐defined $\bar\lambda$ in airborne phases.
\end{remark}

\begin{remark}[Ballistic motion]
When all ends lose contact ($\lambda_l=0$ for all $l$), one finds
\[
\ddot{\mathbf{p}} = \epsilon^2\,\mathbf{p} - \mathbf{g} \approx -\,\mathbf{g},
\quad
\dot{\mathbf{L}} = 0,
\]
recovering the usual ballistic CoM motion and conservation of angular momentum.
\end{remark}

\begin{remark}[Relation to existing models]
Stiffness‐based (or force‐to‐point) parametrization has appeared before, but typically only at the \emph{total} wrench level.  Here we assign a separate $\lambda_l,\mathbf{r}_l,\mathbf{\hat{\boldsymbol{\eta}}}_l$ to each end, which yields a unified multi‐contact description.  The classical CoP and (e)CMP emerge naturally as $\mathbf{\bar{p}}$ and $\mathbf{\bar{p}}+\mathbf{\bar{r}}$, respectively.
\end{remark}

\subsection{Closed‐Form Solutions and Discrete‐Time Equations}

We subdivide the time horizon $[0,T]$ into $N$ consecutive intervals
\[
[t_k,\,t_{k+1}],\quad k=0,1,\dots,N-1,\qquad
t_{k+1}=t_k+\tau_k.
\]
We assume that \emph{contact states} (i.e.\ which ends are in contact) change only at the boundaries $t_k$.  Moreover, we apply a \emph{zero‐order hold} on the stiffness‐based parameters
\[
\{\lambda_{l}(t),\,\mathbf{r}_{l}(t),\,\mathbf{\hat{\boldsymbol{\eta}}}_{l}(t)\}
\quad\mapsto\quad
\{\lambda_{l,k},\,\mathbf{r}_{l,k},\,\mathbf{\hat{\boldsymbol{\eta}}}_{l,k}\}
\quad\text{for }t\in[t_k,t_{k+1}),
\]
meaning that each parameter is held constant over the interval.

\medskip
\paragraph{\ State at the beginning of interval $k$.}
Let
\[
\mathbf{p}_k = \mathbf{p}(t_k),\quad
\mathbf{v}_k = \dot{\mathbf{p}}(t_k),\quad
\mathbf{L}_k = \mathbf{L}(t_k),
\]
and compute the aggregated quantities
\[
\bar\lambda_k
=\sqrt{\sum_{l=1}^{n_e}\lambda_{l,k}^2+\epsilon^2},\quad
\mathbf{\bar{p}}_k
=\frac{\sum_{l=1}^{n_e}\lambda_{l,k}^2\,\mathbf{p}_{l,k}+\mathbf{g}}{\bar\lambda_k^2},\quad
\mathbf{\bar{r}}_k
=\frac{\sum_{l=1}^{n_e}\lambda_{l,k}^2\,\mathbf{r}_{l,k}}{\bar\lambda_k^2},
\]
\[
\mathbf{\bar{\boldsymbol{\eta}}}_k
=\bar\lambda_k^2\,(\mathbf{\bar{p}}_k\times\mathbf{\bar{r}}_k)
\;+\;\sum_{l=1}^{n_e}\lambda_{l,k}^2\bigl(\mathbf{\hat{\boldsymbol{\eta}}}_{l,k}-\mathbf{p}_{l,k}\times \mathbf{r}_{l,k}\bigr).
\]

\paragraph{\ Analytical solution on $[t_k,t_{k+1}]$.}
With $\bar\lambda_k,\mathbf{\bar{p}}_k,\mathbf{\bar{r}}_k,\mathbf{\bar{\boldsymbol{\eta}}}_k$ constant, the CoM‐dynamics 
\[
\ddot{\mathbf{p}} = \bar\lambda_k^2\bigl(\mathbf{p} -(\mathbf{\bar{p}}_k+\mathbf{\bar{r}}_k)\bigr)
\]
is a linear second‐order ODE whose homogeneous+particular solution reads
\[
\mathbf{p}(t)
= (\mathbf{\bar{p}}_k+\mathbf{\bar{r}}_k)
\;+\;C_k(\Delta t)\,\bigl(\mathbf{p}_k -(\mathbf{\bar{p}}_k+\mathbf{\bar{r}}_k)\bigr)
\;+\;\frac{S_k(\Delta t)}{\bar\lambda_k}\,\mathbf{v}_k,
\tag{7a}
\]
\[
\mathbf{v}(t)
=\dot{\mathbf{p}}(t)
=\bar\lambda_k\,S_k(\Delta t)\,\bigl(\mathbf{p}_k -(\mathbf{\bar{p}}_k+\mathbf{\bar{r}}_k)\bigr)
\;+\;C_k(\Delta t)\,\mathbf{v}_k,
\tag{7b}
\]
where $\Delta t = t - t_k$ and
\[
C_k(\Delta t) = \cosh\bigl(\bar\lambda_k\,\Delta t\bigr),\qquad
S_k(\Delta t) = \sinh\bigl(\bar\lambda_k\,\Delta t\bigr).
\]
Finally, substituting into the angular‐momentum equation
\[
\dot{\mathbf{L}} = m\bigl(\ddot{\mathbf{p}}\times \mathbf{\bar{r}}_k + \mathbf{\bar{\boldsymbol{\eta}}}_k\bigr)
\]
and integrating from $t_k$ to $t$ gives
\[
\mathbf{L}(t)
= \mathbf{L}_k
\;+\;m\Bigl(\bigl(\mathbf{v}(t)-\mathbf{v}_k\bigr)\times \mathbf{\bar{r}}_k
   \;+\;(t-t_k)\,\mathbf{\bar{\boldsymbol{\eta}}}_k\Bigr).
\tag{7c}
\]

\medskip
\begin{remark}[Zero‐Order Hold]
A zero‐order hold means we approximate time‐varying parameters by piecewise–constant values on each interval.  This yields closed‐form expressions above, at the cost of not capturing high‐frequency parameter variations.
\end{remark}

\bigskip
\subsection{Integration of Base‐Link Rotation}

The centroidal state $(\mathbf{p},\dot{\mathbf{p}},\mathbf{L})$ does not specify the \emph{orientation} $\mathbf{R}(t)\in SO(3)$ or the \emph{base‐link} angular velocity $\boldsymbol{\omega}(t)\in\Bbb R^3$.  In a multi‐body system one shows
\[
\mathbf{L} \;=\; \underbrace{\mathbf{R}\,I\,\mathbf{R}^\top}_{\displaystyle I_{\rm sys}(\mathbf{R})}\,\boldsymbol{\omega}
\;+\;\mathbf{R}\,\mathbf{\hat{L}},
\tag{8}
\]
where
\begin{itemize}
  \item $I\in\Bbb R^{3\times3}$ is the composite inertia in the base‐link frame,
  \item $\mathbf{\hat{L}}$ is the angular momentum about the base link due to internal motions.
\end{itemize}
If we fix reference values $I_{\rm ref},\,\mathbf{\hat{L}}_{\rm ref}$ (e.g.\ from a nominal whole‐body motion), we solve for
\[
\boldsymbol{\omega}(t)
= \;I_{\rm sys}(\mathbf{R})^{-1}\bigl(\mathbf{L} - \mathbf{R}\,\mathbf{\hat{L}}_{\rm ref}\bigr)
\approx \mathbf{R}\,I_{\rm ref}^{-1}\bigl(\mathbf{R}^\top \mathbf{L} - \mathbf{\hat{L}}_{\rm ref}\bigr).
\tag{9}
\]

\paragraph{Discrete quaternion update.}
Let $\mathbf{q}_k\in\mathbb{H}$ be the unit‐quaternion representing $\mathbf{R}(t_k)$.
Over $[t_k,t_{k+1}]$ we subdivide into $n_{\rm div}$ equal steps
\[
t_k = t'_0 < \cdots < t'_i < \cdots < t'_{n_{\rm div}} = t_{k+1},\qquad
\tau_k' = \frac{\tau_k}{n_{\rm div}},\quad
t'_i = t_k + i\,\tau_k'.
\]
At each substep we assume $\boldsymbol{\omega}$ nearly constant and update
\[
\mathbf{q}_{k+1}
= \underbrace{\mathbf{q}\bigl(\boldsymbol{\omega}(t'_{n_{\rm div}})\,\tau_k'\bigr)}
_{\text{quat. for small rotation}}
\;\cdots\;
\mathbf{q}\bigl(\boldsymbol{\omega}(t'_1)\,\tau_k'\bigr)\;\cdot\;
\mathbf{q}\bigl(\boldsymbol{\omega}(t'_0)\,\tau_k'\bigr)\;\cdot\;\mathbf{q}_k,
\tag{10}
\]
where $\mathbf{q}(\boldsymbol{\theta})$ is the unit quaternion corresponding to the axis‐angle $\boldsymbol{\theta}\in\Bbb R^3$.  The designer chooses $n_{\rm div}$ to balance integration accuracy against computational cost of gradient evaluation in trajectory optimization.
\end{sloppypar}

\end{document}