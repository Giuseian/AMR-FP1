\documentclass[12pt,a4paper,oneside]{article}
\usepackage{graphicx}
\usepackage{titlepic}
\usepackage[utf8]{inputenc}
\usepackage[left=1.1in,right=1.1in, top=1in, bottom= 1in]{geometry}
\usepackage{amsthm}
\usepackage{amsfonts}
\usepackage{amssymb}
\usepackage{amsmath}
\usepackage{fancyhdr}
\usepackage{hyperref}
\usepackage{etoolbox}
\usepackage[nottoc]{tocbibind}
\usepackage{appendix}
\usepackage{multicol}
\usepackage{leftidx}
\graphicspath{{figures/}}
\usepackage{ragged2e}
\usepackage{mathtools}
\usepackage{units}
\usepackage{float}
\usepackage{subcaption}
\usepackage{commath}
\usepackage{comment}




% proposed method stuff 
\theoremstyle{definition}
\newtheorem{definition}{Definition}[section]

\newtheorem{theorem}{Theorem}

\theoremstyle{remark} % <-- ADD THIS
\newtheorem{remark}{Remark} % <-- ADD THIS


% simulation packages 
\usepackage{algorithm}
\usepackage{algpseudocode}
\usepackage{amsmath}
\usepackage{listings}
\usepackage{xcolor}
\usepackage{subcaption} 
\usepackage{booktabs}
\usepackage{media9}
 
%\theoremstyle{definition}
%\newtheorem{definition}{Definition}[section]

%\newtheorem{theorem}{Theorem}

\usepackage[none]{hyphenat} % Avoids to go out of margin

\usepackage{subfiles}

% --------------------------------------------- %


\title{Trajectory Generation for Legged Robots Based on a  Closed-Form Solution of Centroidal Dynamics}	                                    % Title
\author{Giuseppina Iannotti Francesco Danese Alessia Pontiggia}
\date{\today}									    % Date


% Font size of figure smaller than normal size:
\usepackage{caption}
\captionsetup[figure]{font=small}
%\captionsetup[table]{font=small}

%\usepackage{setspace} % double spacing
\linespread{1.2}

\makeatletter
\let\thetitle\@title
\let\theauthor\@author
\let\thedate\@date
\makeatother

\begin{document}

\begin{titlepage}
	\centering
    \vspace*{0.5 cm}
    \includegraphics[scale = 0.75]{figures/SapienzaLogo.pdf}\\[1.0 cm]	% University Logo

    \vspace*{-0.4cm}
    \textsc{\large Department of Computer Control and Management Engineering}\\[2.0 cm]	% Department Name
    \vspace*{1cm}

    { \fontsize{20.74pt}{18.5pt}\selectfont\bfseries \thetitle \par } % Title

    \vspace*{0.25cm}
    \textsc{\Large Autonomous and Mobile Robotics}\\[0.5 cm] % Course Name

    \vspace*{2.6cm}
	\begin{minipage}[t]{0.45\textwidth}
		\begin{flushleft} \large
			\textbf{Professor:}\\
			Prof. Giuseppe Oriolo\\
			\textbf{Supervisor:}\\
			Nicola Scianca
		\end{flushleft}
	\end{minipage}
	\hspace{1cm}
	\begin{minipage}[t]{0.45\textwidth}
		\begin{flushleft} \large
			\textbf{Students:}\\
			Francesco Danese 1926188
			Giuseppina Iannotti 1938436
			Alessia Pontiggia 1892079
			%\theauthor
		\end{flushleft}
	\end{minipage}

    \vspace{3cm}

    \rule{\linewidth}{0.2 mm} \\[0.3 cm]
    \vspace*{-0.2cm}
    Academic Year 2024/2025
\end{titlepage}


\newpage
\tableofcontents
\newpage
\begin{abstract}
\begin{sloppypar}
\noindent
Recent advances in robotics have led to increasing interest in dynamic motion planning for humanoid and legged robots, which typically have many degrees of freedom. However, generating feasible trajectories in real time using full dynamic models remains computationally challenging. To address this, reduced-order models are often employed, although they can suffer from limited accuracy in complex scenarios. This project builds on the method proposed by Tazaki et al. \cite{tazaki2024trajectory}, which introduces a stiffness-based parametrization of contact wrenches to derive closed-form centroidal dynamics and improve the efficiency of trajectory optimization. The resulting model, known as Stiffness-Based Centroidal Dynamics (SBCD), describes both translational and rotational motion while maintaining a compact and analytically tractable form. This structure makes it particularly well-suited for long-horizon planning and real-time control. The goal of this work is to implement and evaluate the SBCD framework through the generation of reference trajectories for walking and standing tasks. The proposed approach is tested in simulation and assessed using quantitative metrics and visual analysis.
\end{sloppypar}
\end{abstract}
\newpage
\subfile{chapters/Introduction}
\clearpage
\subfile{chapters/Related Works}
\clearpage
\subfile{chapters/Proposed Method_ Stiffness-Based Centroidal Dynamics}
\clearpage
\subfile{chapters/Formulation and Trajectory Optimization Problem}
\clearpage
\subfile{chapters/Simulation and Results}
\clearpage
\subfile{chapters/Conclusion}
\clearpage

\bibliographystyle{unsrt}
\bibliography{ref}

\end{document}