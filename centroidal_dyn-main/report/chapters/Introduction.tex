\documentclass[main.tex]{subfiles}

\begin{document}

\section{Introduction}
\label{sec:introduction}
Humanoid and legged robots have seen rapid development in recent years, leading to their adoption in areas such as logistics, surveillance, and social interaction. Their many degrees of freedom allow them to perform versatile and dynamic movements, but this complexity makes real-time trajectory generation using full dynamic models computationally demanding. To reduce computational complexity, researchers have increasingly relied on reduced-order (template) models that approximate the system’s key dynamics. These models balance efficiency and accuracy, and are commonly combined with full models in trajectory optimization and model predictive control. \\
Among the most well-known reduced-order models are the Linear Inverted Pendulum (LIPM) \cite{kajita1991study} and its variant, the Variable-Height LIPM \cite{caron2020biped}, which assume simplified linear dynamics, making them suitable for structured environments but inadequate for dynamic or uneven terrains. On the other hand, models like the Spring-Loaded Inverted Pendulum (SLIP) \cite{poulakakis2009spring} more accurately capture the underlying physical dynam-\\ics, but lack of a closed form solutions, making optimization more complex. As a result, trajectory planning with these models often leads to large-scale problems, limiting their real-time applicability.\\
To address the trade-off between computational efficiency and model expressiveness, Tazaki et al.~\cite{tazaki2024trajectory} propose a novel approach based on a stiffness-based parametrization of contact wrenches. This formulation simplifies centroidal dynamics and enables more efficient trajectory optimization over extended time horizons. The main contributions of the work are as follows. First, the authors introduce the \textbf{Stiffness-Based Centroi-\\} \textbf{dal Dynamics (SBCD)} model, which captures both translational and rotational motion components. Second, they derive \textbf{closed-form expressions} for these dynam-\\ics using the proposed contact wrench parametrization. Third, a \textbf{trajectory optimi-\\} \textbf{zation framework} is developed, incorporating task objectives, physical constraints, and contact-related cost terms to support both static and dynamic behaviors.\\
This project aims to implement the proposed SBCD model and to reproduce the results presented in the original work for both walking and static balance tasks. It includes the development of reference generation algorithms tailored to these specific tasks. The algorithms are evaluated through simulation, and their performance is assessed using quantitative metrics, plots, and tables.\\
\newline
The organization of this report is as follows.In Section \ref{sec:relatedworks}, related works in reduced-order modeling and trajectory optimization are reviewed. In Section \ref{sec:proposedmethods}, the derivation of the proposed SBCD model is detailed. In Section \ref{sec:formulation}, the trajectory optimization problem is formulated and the adopted solution approach is presented. Concluding, in Section \ref{sec:simulation}, experimental results are shown, including algorithmic implementation and demonstrations of standing and walking tasks. 
\end{document}