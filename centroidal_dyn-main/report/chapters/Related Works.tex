\documentclass[main.tex]{subfiles}

\begin{document}

\section{Related Works}\label{sec:relatedworks}
%The study of Centroidal Dynamics plays a fundamental role in understanding and controlling humanoid robots \cite{CD}. Centroidal Dynamics refers to the dynamics of a robot projected onto its Center of Mass (CoM)—a uniquely critical point that represents the effective location of the robot’s entire mass and the point through which the resultant gravitational force acts. Despite the inherent complexity of humanoid robots, often modeled as multi-link chains with high-dimensional and nonlinear dynamics, the behavior of their CoM follows simpler, more intuitive laws. By focusing on the motion and forces at the CoM, it is possible to capture the essential aspects of a robot's global dynamics without needing to model every individual joint and link in fine detail. This abstraction not only reduces complexity but also provides a powerful framework for planning, control, and stability analysis, ultimately enabling the design of more robust, agile, and human-like behaviors.
Understanding Centroidal Dynamics \cite{orin2013centroidal} is essential for controlling humanoid robots. Centroidal Dynamics \cite{orin2013centroidal} refers to the dynamics of a robot projected onto its Center of Mass (CoM),  which represents the average position of all its mass and serves as the point where the overall gravitational force can be considered to act. Although humanoid robots are complex systems with high-dimensional, nonlinear dynamics, the motion of their Center of Mass (CoM) can often be described using simpler, more intuitive models. Focusing on the CoM allows the global dynamics of the robot to be captured without modeling each joint and link individually, significantly reducing system complexity. This approach provides an effective framework for planning, control, and stability, enabling the development of robots with robust, agile, and human-like behavior. Thanks to these advantages, many reduced order models have been developed by simplifying CD in different ways.\\
\paragraph{Simplified Centroidal Models} One of the foundational reduced-order models in humanoid locomotion is the Linear Inverted Pendulum Model (LIPM) \cite{kajita1991study}. This model treats the robot as a point mass and simplifies the equations of motion by assuming a constant height of the Center of Mass and neglecting angular momentum. These assumptions linearize the dynamics and make the model especially suitable for fast and efficient planning and control in legged robots. The model ensures stability by maintaining the Zero Moment Point (ZMP) within the support polygon defined by the feet. To improve upon LIPM and allow for more dynamic behaviors, the Variable-Height Linear Inverted Pendulum Model (VH-LIPM) \cite{caron2020biped} was introduced. This model integrates a linear feedback controller that aligns with the 3D Divergent Component of Motion (DCM) \cite{englsberger2015three} under feasible conditions and leverages vertical CoM variations when the ZMP nears the edge of the support region. Another widely adopted model is the Spring-Loaded Inverted Pendulum (SLIP), which assumes a compliant leg structure and is often used to replicate running dynamics \cite{full1999templates, holmes2006dynamics}. An extension of this, the Asymmetric SLIP (ASLIP) model \cite{poulakakis2009spring}, combines the flexibility of SLIP with the formal guarantees of Hybrid Zero Dynamics (HZD) \cite{westervelt2003hybrid} control theory to produce stable running motions. The ASLIP includes torso dynamics that are nontrivially coupled with leg motion, further enhancing its realism.
\paragraph{Integration of Trajectory Optimization} Reduced-order models like LIPM, SLIP and VH-LIPM are commonly embedded into trajectory optimization (TO) frameworks. TO involves computing optimal motion plans by minimizing a cost function subject to dynamic and physical constraints. In these setups, reduced models serve as simplified system representations within these optimization problems. Approaches to TO for CoM trajectories vary. Some use LIPM as the state equation \cite{englsberger2017smooth, kamioka2018simultaneous}, defining desired ZMP as a cost \cite{tedrake2015closed} and enforcing stability through ZMP constraints \cite{ajita2003biped, murooka2022centroidal}. When both CoM movement and base link rotation are considered, centroidal dynamics is used either as a constraint or state equation \cite{kuindersma2016optimization, rossini2023real}, with stability enforced via support criteria like the ZMP region or the Centroidal Wrench Cone (CWC) \cite{ding2021nonlinear, dai2016planning}.
\begin{sloppypar}
\paragraph{Limitations and Solutions} Despite their flexibility, numerical trajectory optimization techniques often require small time steps for accurate integration, leading to a large number of decision variables and increased computational cost. Additionally, these methods usually guarantee feasibility only at discrete time points, leaving feasibility in between unverified. Some attempts have been made to mitigate these issues by linearizing centroidal dynamics—typically by neglecting rotational effects and fixing the CoM height \cite{audren2014model}. While these simplifications can be effective in conventional walking scenarios, they fall short when tackling more dynamic tasks like multi-contact planning or acrobatic maneuvers. To address the limitations of discretization-heavy methods, researchers have explored closed-form solutions of centroidal dynamics. A common method involves treating contact wrenches as piecewise constant (zero-order hold), which enables larger integration intervals. However, this often induces undesirable angular momentum fluctuations unless extremely short time steps are used. In \cite{tazaki2022fast}, the multi-contact (mc-) LIPM is proposed, expressing contact forces as functions of stiffness and the displacement between contact points and the CoM. This method allows for larger integration steps—potentially spanning entire contact phases—without significant angular momentum disturbances. \\
\\
To overcome these drawbacks, a novel stiffness-based method is introduced \cite{tazaki2024trajectory}. This method derives closed-form solutions for centroidal dynamics by parameterizing contact wrenches through stiffness models. These analytical solutions are then integrated into trajectory optimization, enabling the generation of longer, dynamically feasible trajectories with significantly fewer decision variables.
\end{sloppypar}


%Thanks to these advantages, many reduced order models have been developed by simplyfing CD in different ways 



%Thanks to these advantages, many reduced order models and control algorithms have been developed that incorporate the CoM and Centroidal Dynamics as key components. Among the earliest and most influential models is the Linear Inverted Pendulum Model (LIPM) \cite{LIPM}, a point-mass approximation that simplifies the robot’s dynamics by making two assumptions: (1) there is no angular momentum about the CoM, and (2) the CoM maintains a constant height. These assumptions allow the otherwise nonlinear centroidal dynamics to be linearized, greatly simplifying trajectory planning. Stability within this framework is maintained by ensuring that the Zero Moment Point (ZMP)—the point where the resultant ground reaction force would act—remains inside the support polygon formed by the robot’s feet.

%To extend the capabilities of LIPM, the Variable-Height Linear Inverted Pendulum Model (VH-LIPM) \cite{VHLIPM} was introduced. This model relaxes the constant height assumption, allowing vertical CoM motion, thereby enabling more dynamic and versatile movements such as stepping on uneven terrain or climbing.

%Another important model is the Spring-Loaded Inverted Pendulum (SLIP) \cite{SLIP}, which represents the CoM supported by a massless, compliant leg modeled as a spring. Unlike the rigid models, SLIP captures the energy storage and return mechanisms crucial for running, hopping, and other dynamic gaits. The ability of SLIP to express vertical oscillations of the CoM makes it essential for modeling and analyzing dynamic, non-walking locomotion.

%These reduced order models have been extensively used in trajectory optimization approaches for motion planning. Trajectory optimization involves defining a motion task through state variables, control inputs, cost functions, and constraints, and then solving for the trajectory that optimizes performance while respecting dynamic feasibility. The reduced models serve as simplified system representations within these optimization problems. For instance, LIPM-based trajectory planners might define the desired ZMP trajectory as a cost term or enforce ZMP constraints to guarantee stability. More advanced formulations integrate full centroidal dynamics, incorporating both CoM motion and base orientation, with stability constraints expressed through support polygons or centroidal wrench cones.

%However, numerical trajectory optimization approaches often require fine time discretization—small time steps—to maintain solution accuracy, leading to a high number of decision variables. Moreover, feasibility is typically ensured only at discrete time nodes, and ensuring feasibility between these nodes remains a challenge.

%\paragraph{Integration of Trajectory Optimization} If a closed-form solution to the dynamic model is available, it enables trajectory generation using far fewer decision variables while ensuring feasibility at all times, not just at discrete points. Some methods have attempted to derive closed-form or analytical solutions for centroidal dynamics, often through simplifying assumptions such as constant CoM height or neglecting vertical-axis rotation. While these approaches are beneficial for efficiency, they limit the applicability of the resulting motions, particularly for more complex tasks like multi-contact planning or acrobatics. One technique for analytical integration involves assuming constant contact wrenches over finite time intervals (zero-order hold), but this introduces unwanted angular momentum variations unless extremely small time steps are used. To address this limitation, the multi-contact Linear Inverted Pendulum Model (mc-LIPM) was proposed, modeling contact forces through a stiffness-displacement relationship between the CoM and the contact points. This method allows for larger integration steps—potentially spanning entire contact phases—without significant angular momentum disturbances. Nevertheless, while mc-LIPM improves over traditional LIPM approaches, it still lacks full expressiveness for modeling complex rotational dynamics. Despite their success, these reduced order models introduce fundamental limitations. LIPM and VH-LIPM are built on strong assumptions that oversimplify the nonlinear nature of real-world dynamics, restricting their use in highly dynamic or irregular environments. SLIP preserves more physical realism but lacks a closed-form solution, complicating its analytical handling in optimization frameworks. Moreover, trajectory optimization methods based on these models often lead to large-scale optimization problems that are computationally expensive, challenging their application to real-time control.
%\paragraph{Limitations and Solutions} These simplifications may be tolerable for basic locomotion tasks such as walking or slow stepping, but they become inadequate for advanced applications like multi-contact maneuvers, dynamic manipulation, or acrobatic motions, where high-fidelity and continuous-time feasibility are crucial. Additionally, the lack of closed-form solutions exacerbates the computational burden, making fast online planning difficult. To overcome these challenges, a novel stiffness-based method is introduced. This approach provides a closed-form solution to the trajectory optimization problem by modeling the interaction forces through stiffness relationships. It enables the generation of long, dynamically feasible trajectories with far fewer decision variables, offering better computational efficiency and making real-time, complex motion generation more practical and reliable.
\end{document}