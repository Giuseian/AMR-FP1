\documentclass[main.tex]{subfiles}
\usepackage{graphicx} % Required for inserting images
\usepackage{algorithm}
\usepackage{algpseudocode}
\usepackage{amsmath}
\usepackage{listings}
\usepackage{xcolor}
\usepackage{subcaption} 
\usepackage{booktabs}
\usepackage{media9}

\begin{document}

\section{Simulation and Results}\label{sec:simulation}

This section aims to visualize the results obtained with the proposed method. In the following, there is first defined the trajectory used for the task reference. Then, there are shown plots, tables and simulations.
\\Before defining the trajectories implemented, let's denote with:
\begin{itemize}
    \item 0, when the foot is in contact with the surface
    \item -, when the foot is off contact with the surface
\end{itemize}
The trajectories are defined with respect to contact point, not time.


\subsection{Trajectory Generation}
\subsubsection*{Still Task}
In this task the robot should not move at any instant of time and thus keeping the same initial position. 
The contact sequence used is reported in Table ~\ref{tab:cs_still} where the first row concerns the right foot and the second row the left foot.
\begin{table}[H]
\label{tab:cs_still}
\centering
\begin{tabular}{|c|c|l|}
\hline
\textbf{Task} & \textbf{N} & \textbf{Contact Sequence} \\
\hline
Still & 4 & 
\begin{tabular}[c]{@{}l@{}} 
0\,0\,0\,0 \\
0\,0\,0\,0
\end{tabular} \\
\hline
\end{tabular}
\caption{Contact sequence for still task}
\end{table}

The trajectory is simply defined as follows:
\begin{algorithm}[H]
\caption{Reference Trajectory Initialization and Update}
\begin{algorithmic}[1]
\State Initialize $X_{\text{ref}} \in \mathbb{R}^{28 \times (N+1)}$, $U_{\text{ref}} \in \mathbb{R}^{27 \times N}$
\State Set initial CoM state and feet state in $X_{\text{ref}}$
\State $\text{time} \gets 0$

\For{$t = 0$ to $N-1$}
    \State Read current contact state from $\sigma$
    \State Set phase duration $\tau$ and contact force gains $\lambda$
    \State Update $U_{\text{ref}}$ with the above
    \State Set CoM and feet position, velocities and orientation as the initial state values 
    \State time = time + phase duration
    \State Update $X_{\text{ref}}(t+1)$ with the above
\EndFor
\State \Return $X_{\text{ref}}, U_{\text{ref}}$
\end{algorithmic}
\end{algorithm}


\subsubsection*{Walking Task}
For the walking task, the trajectory used relies on the following concepts:
during the double-contact phases, in which both feet are on the ground ([0,0]), the COM does not move forward; it moves along the X-axis only during single-support phases (a step), namely when right/left foot is lifting ([-, 0] or [0, -]). The Z (height) of each foot is always zero since at the beginning of each phase the feet are on the ground, while the X increases in an alternating manner depending on which foot was lifted in the previous phase. Also, the feet moves simulating a real walk, where the foot that is moving "surpasses" the one in contact, touching the ground beyond it. We also considered an oscillation of the CoM around the Y-axis, as well as minor modifications for the first step, when the robot starts with the feet next to each other. Each velocity component was crafted accordingly. Lastly, each foot is interpolated in the intra-phase height position by following a parabolic profile that starts and arrives at the corresponding positions given by the phases solutions, instead of using the fixed inputs (zero-hold). The contact sequence used is reported in Table ~\ref{tab:cs} where the first row concerns the right foot and the second row the left foot.
\begin{table}[H]
\label{tab:cs}
\centering
\begin{tabular}{|c|c|l|}
\hline
\textbf{Task} & \textbf{N} & \textbf{Contact Sequence} \\
\hline
Walk & 24 & 
\begin{tabular}[c]{@{}l@{}} 
0\,0\,0\,-\,0\,0\,0\,-\,0\,0\,0\,-\,0\,0\,0\,-\,0\,0\,0\,-\,0\ \\
0\,-\,0\,0\,0\,-\,0\,0\,0\,-\,0\,0\,0\,-\,0\,0\,0\,-\,0\,0\,0 
\end{tabular} \\
\hline
\end{tabular}
\caption{Contact sequence for walking task}
\end{table}
A general outline of the algorithm used for this task is shown below:


\begin{algorithm}[H]
\caption{Reference Trajectory Initialization and Update}
\begin{algorithmic}[1]
\State Initialize $X_{\text{ref}} \in \mathbb{R}^{28 \times (N+1)}$, $U_{\text{ref}} \in \mathbb{R}^{27 \times N}$
\State Set initial CoM state and feet state in $X_{\text{ref}}$
\State $\text{time} \gets 0$

\For{$t = 0$ to $N-1$}
    \State Read current contact state from $\sigma$
    \State Set phase duration $\tau$ and contact force gains $\lambda$
    \State Update $U_{\text{ref}}$ with the above
    \State Set CoM velocity and feet velocities, according to which foot is about to move
    \State Set CoM and feet positions based on velocity and duration $(p = v*\tau)$
    \State time = time + phase duration
    \State Update $X_{\text{ref}}(t+1)$ with the above
\EndFor

\State \Return $X_{\text{ref}}, U_{\text{ref}}$
\end{algorithmic}
\end{algorithm}


\subsection{Solutions}
Solutions of those problems and tasks are tuples (X, U) where X is the state vector with dimension N , and U is the input vector with dimension N-1.
\\The solution is computed with ipopt solver from Casadi optimizer. Both the reference and solution are contact-dependent, meaning that X and U contain values for each contact phase, thus to find the time dependent solution we used the dynamics. 

\subsection{Still Task: results}
In this case the state vector has dimension 3, and the input vector has dimension 2.
\subsubsection*{CoM Plots}
The following plots report the CoM trajectory in time. 
The alternation between light and dark grey on the background suggests the switch between phases.
As plots suggest there's no significant motion in any direction.
\begin{figure}[H]
    \centering
    \begin{subfigure}[b]{0.45\textwidth}
        \centering
        \includegraphics[width=\textwidth]{figures/CoM x Trajectory still.png}
        \caption{Com X Trajectory 1}
        \label{fig:sub1_still}
    \end{subfigure}
    \hfill
    \begin{subfigure}[b]{0.45\textwidth}
        \centering
        \includegraphics[width=\textwidth]{figures/CoM y Trajectory still.png}
        \caption{Com Y Trajectory 1}
        \label{fig:sub2_still}
    \end{subfigure}
    \hfill
    \begin{subfigure}[b]{0.45\textwidth}
        \centering
        \includegraphics[width=\textwidth]{figures/CoM z Trajectory still.png}
        \caption{Com Z Trajectory 1}
        \label{fig:sub3_still}
    \end{subfigure}
    \caption{Com Trajectory}
    \label{fig:threeimages_still}
\end{figure}

\begin{figure}[htbp]
    \centering
    \includegraphics[width=0.6\textwidth]{figures/Feet_z still.png}
    \caption{Feet along Z (scale is 1e-8)}
    \label{fig:feet_still}
\end{figure}


\subsubsection*{State and Input Contact Values}
In the following plots, the main components of the state and input vectors solutions are compared with the reference at each contact step.
\begin{figure}[htbp]
    \centering
    \includegraphics[width=0.6\textwidth]{figures/contact_x_still.png}
    \caption{Trajectory vs Reference: state vector}
    \label{fig:contact_x_still}
\end{figure}

\subsubsection*{Contact Forces}
Another important dynamic aspect regards forces. 
In the input vector, the contact wrench that describes all the mechanical influence that a contact point (like a foot or hand) exerts on the robot or vice versa.
To balance dynamic laws, along the Z-axis the environment should exert a reaction force equal to the gravity factor multiplied by the mass of the robot. In the table ~\ref{tab:contact_forces_still}  below there are listed the contact forces exerting from the environment to both feet. As the table shows, when both feet are on the ground the gravity force is equally distributed on left and right foot.
\begin{table}[H]
\label{tab:contact_forces_still}
\centering
\begin{tabular}{ccc}
\toprule
Right Foot Z & Left Foot Z & $\Sigma_L^k$ \\
\midrule
49.0644 & 49.0531 & [0., 0.] \\
48.8976 & 49.9925 & [0., 0.] \\
48.0847 & 49.0895 & [0., 0.] \\
\bottomrule
\end{tabular}
\caption{Z-axis gravity force and $\Sigma_L^k$ values}
\end{table}

Components of contact wrench, rotational and translational forces, are graphically expressed in the following plots:
\begin{figure}[htbp]
    \centering
    \includegraphics[width=0.8\textwidth]{figures/contact_forces_still.png}
    \caption{Trajectory vs Reference: Forces}
    \label{fig:contact_forces_still}
\end{figure}

\subsection{Walking Task: results}
In this case the state vector has dimension 24 and the input vector has dimension 23.

\subsubsection*{CoM Plots}
The following plots report the CoM trajectory in time. 
The alternation between light and dark grey on the background suggests the switch between phases.
As Fig. ~\ref{fig:sub1} shows, the CoM exhibits linear motion in the x-direction.
Along the y-axis ~\ref{fig:sub2} the CoM goes towards the foot staying on the ground when lifting the other. Let's consider the first two contacts: in the first one, the robot is in double support (both feet are on the ground), while the second one expects the left foot to raise. Infact, the CoM at the end of the first phase has moved towards the right foot. This mechanism is repeated through all the alternances between double support and single support.
In the z-direction ~\ref{fig:sub3}, the CoM goes up and down in a small range of motion.
\begin{figure}[H]
    \centering
    \begin{subfigure}[b]{0.45\textwidth}
        \centering
        \includegraphics[width=\textwidth]{figures/CoM x Trajectory walking.png}
        \caption{Com X Trajectory 1}
        \label{fig:sub1_walking}
    \end{subfigure}
    \hfill
    \begin{subfigure}[b]{0.45\textwidth}
        \centering
        \includegraphics[width=\textwidth]{figures/CoM y Trajectory walking.png}
        \caption{Com Y Trajectory 1}
        \label{fig:sub2_walking}
    \end{subfigure}
    \hfill
    \begin{subfigure}[b]{0.45\textwidth}
        \centering
        \includegraphics[width=\textwidth]{figures/CoM z Trajectory walking.png}
        \caption{Com Z Trajectory 1}
        \label{fig:sub3_walking}
    \end{subfigure}
    \caption{Com Trajectory}
    \label{fig:threeimages_walking}
\end{figure}
Furthermore, the image below represent the foot position along the Z-axis in time. As one can see, they follow a parabolic profile.  
\begin{figure}[htbp]
    \centering
    \includegraphics[width=0.6\textwidth]{figures/Feet_z walking.png}
    \caption{Feet along Z}
    \label{fig:feet_walking}
\end{figure}


\subsubsection*{State and Input Contact Values}
In the following plots, the main components of the state and input vectors solutions are compared with the reference at each contact step.
\begin{figure}[htbp]
    \centering
    \includegraphics[width=0.6\textwidth]{figures/contact_x_walking.png}
    \caption{Trajectory vs Reference: state vector}
    \label{fig:contact_x_walking}
\end{figure}

\subsubsection*{Contact Forces}
Another important dynamic aspect regards forces. 
In the input vector, the contact wrench that describes all the mechanical influence that a contact point (like a foot or hand) exerts on the robot or vice versa.
To balance dynamic laws, along the Z-axis the environment should exert a reaction force equal to the gravity factor multiplied by the mass of the robot. In the table ~\ref{tab:contact_forces_walking}  below there are listed the contact forces exerting from the environment to both feet. As the table shows, when both feet are on the ground the gravity force is equally distributed on left and right foot, while when one foot lifts the gravity force acts only on the foot staying on the ground.
\begin{table}[H]
\label{tab:contact_forces_walking}
\centering
\begin{tabular}{ccc}
\toprule
Right Foot Z & Left Foot Z & $\Sigma_L^k$ \\
\midrule
49.0644 & 49.0531 & [0., 0.] \\
97.9551 & 0 & [0., -] \\
48.8973 & 49.65 & [0., 0.] \\
0 & 97.4904 & [-, 0.] \\
49.6528 & 49.1563 & [0., 0.] \\
97.3081 & 0 & [0., -] \\
49.2399 & 49.7241 & [0., 0.] \\
0 & 97.2091 & [-, 0.] \\
49.7388 & 49.293 & [0., 0.] \\
97.1669 & 0 & [0., -] \\
49.3007 & 49.7613 & [0., 0.] \\
0 & 97.1486 & [-, 0.] \\
49.762 & 49.31 & [0., 0.] \\
97.1443 & 0 & [0., -] \\
49.3062 & 49.7655 & [0., 0.] \\
0 & 97.1495 & [-, 0.] \\
49.7557 & 49.3045 & [0., 0.] \\
97.1685 & 0 & [0., -] \\
49.2815 & 49.7467 & [0., 0.] \\
0 & 97.216 & [-, 0.] \\
49.7003 & 49.254 & [0., 0.] \\
97.3264 & 0 & [0., -] \\
49.1286 & 49.6485 & [0., 0.] \\
0 & 97.5828 & [-, 0.] \\
\bottomrule
\end{tabular}
\caption{Z-axis gravity force and $\Sigma_L^k$ values}
\end{table}

Components of contact wrench, rotational and translational forces, are graphically expressed in the following plots:
\begin{figure}[htbp]
    \centering
    \includegraphics[width=0.8\textwidth]{figures/contact_forces_walking.png}
    \caption{Trajectory vs Reference: Forces}
    \label{fig:contact_forces_walking}
\end{figure}

\newpage
\subsubsection*{Robot representation}
In this section it is possible to visualize the path followed by the CoM, right and left foot in time:
\begin{figure}[htbp]
    \centering
    \includegraphics[width=0.8\textwidth]{figures/walking.PNG}
    \caption{Walking}
    \label{fig:walking}
\end{figure}
 

\end{document}