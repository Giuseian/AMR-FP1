\documentclass[main.tex]{subfiles}

\begin{document}

\section{Conclusion}\label{sec:conclusion}
\begin{sloppypar}
    In this project, we implemented and evaluated the Stiffness-Based Centroidal Dynamics (SBCD) model proposed by Tazaki, with a focus on trajectory optimization for both static balancing and dynamic walking tasks. 
    Overall, the SBCD framework presents a compelling alternative to existing reduced-order models, offering a balance between physical fidelity and computational tractability that is well-suited for real-time trajectory planning in legged robots.
    Infact, this formulation allows for analytic integration over finite time intervals under zero-order hold assumptions and yields closed-form expressions for CoM trajectory and angular momentum evolution. Furthermore, we outlined a practical method for integrating base-link orientation from the centroidal state using quaternion updates and nominal inertia models. \\This work presented also a comprehensive simulation framework for generating and analyzing motion trajectories in both static and dynamic scenarios for a bipedal robot. Two primary tasks were examined: a still task, emphasizing balance and immobility, and a walking task, showcasing locomotion with dynamically varying contact phases. The trajectory generation methods effectively incorporated contact state information and phase-dependent parameters to produce physically consistent reference motions. 
    \\At the end, simulation results demonstrated that the robot successfully maintained static equilibrium in the still task and achieved stable locomotion in the walking task. For instance, the CoM trajectories profiles and contact forces values confirmed the physical plausibility and effectiveness of the control strategy, while the optimization parameters ensured robust convergence and adherence to realistic constraints. 
    \\One of the main challenges encountered during the implementation was the selection and tuning of optimization parameters and cost function weights. Since the behavior of the trajectory optimizer is highly sensitive to these weights, we observed that small variations could lead to drastically different motion outcomes — ranging from overly conservative to physically unstable behaviors. This tuning process required extensive trial and error, as well as manual adjustments to balance competing objectives such as smooth motion, adherence to physical constraints, and task performance. 
    \\Despite these difficulties, the final implementation demonstrated successful reproduction of the reference walking and standing behaviors presented in the original work. The generated trajectories respected contact constraints and achieved plausible centroidal motion, validating the effectiveness of the SBCD model.
    \\These results validate the proposed approach and establish a foundation for extending the framework to more complex maneuvers and real-world implementation.
    \\Future work could extend this approach by incorporating model uncertainty, testing the framework on hardware, and exploring more complex multi-contact scenarios. Additionally, the integration of learning-based components for adaptation and robustness in unstructured environments represents a promising direction.
\end{sloppypar}
\end{document} 