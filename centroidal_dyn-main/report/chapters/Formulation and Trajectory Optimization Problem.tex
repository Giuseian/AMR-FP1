\documentclass[main.tex]{subfiles}

\begin{document}

\begin{sloppypar}
\section{Trajectory Optimization}
\label{sec:formulation}
\subsection{State Equation}
To formally describe the evolution of the system over discrete time intervals, we define the state and input vectors at each time step \( k \). The state vector \( \mathbf{x}_k \) includes all variables necessary to characterize the system’s configuration and motion, while the control input vector \( \mathbf{u}_k \) is defined according to the stiffness-based control strategy introduced in Section~\ref{sec:proposedmethods}. These vectors are structured as follows:
\begin{equation}
\label{eq:state_input}
\mathbf{x}_k =
\begin{bmatrix}
\mathbf{p}_k \\
\mathbf{q}_k \\
\mathbf{v}_k \\
\mathbf{L}_k \\
t_k \\
\{\mathbf{p}_{l,k}\}_{l=1,\ldots,n_e} \\
\{\mathbf{q}_{l,k}\}_{l=1,\ldots,n_e}
\end{bmatrix}
\hspace{4cm}
\mathbf{u}_k =
\begin{bmatrix}
\tau_k \\
\{\mathbf{v}_{l,k}\}_{l=1}^{n_e} \\
\{\boldsymbol{\omega}_{l,k}\}_{l=1}^{n_e} \\
\{\lambda_{l,k}\}_{l=1}^{n_e} \\
\{\mathbf{r}_{l,k}\}_{l=1}^{n_e} \\
\{\boldsymbol{\eta}_{l,k}\}_{l=1}^{n_e}
\end{bmatrix}
\end{equation}
By defining end-effector velocities as control inputs, contact complementarity can be enforced penalizing motion at contact points through high velocity costs.
\paragraph{Equations Update}We now present the update equations that define the system's evolution over discrete time steps. \\ 
The update of the timestamp is defined as:
\begin{equation}
    t_{k+1} = t_k + \tau_k
\end{equation}
Next, the position and orientation of each end-effector are updated using basic kinematic relations:
\begin{equation}
    \mathbf{p}_{l,k+1} = \mathbf{p}_{l,k} + \mathbf{v}_{l,k} \tau_k
\end{equation}
\begin{equation}
    \mathbf{q}_{l,k+1} = \textit{q}(\boldsymbol{\omega}_{l,k} \tau_k) \cdot \mathbf{q}_{k}
\end{equation}
Here, $\textit{q}(\boldsymbol{\omega}_{l,k} \tau_k)$ denotes the quaternion representing the angular displacement resulting from integrating the angular velocity $\boldsymbol{\omega}_{l,k}$ over the time step $\tau_k$.
Integrating all these elements, the system dynamics can be compactly represented by the following state transition function:
\begin{equation}
    \mathbf{x}_{k+1} = f(\mathbf{x}_k, \mathbf{u}_k)
\end{equation}
\subsection{Formulation of Optimal Control Problem}
In trajectory optimization, the goal is to find a sequence of control inputs that minimize a cost function while satisfying the system dynamics and any constraints. This cost function evaluates the quality of a trajectory and typically consists of multiple terms, each reflecting a specific performance objective.
\paragraph{Task Related Cost}
In trajectory tracking tasks, it is important for the system to follow a planned or reference trajectory as closely as possible. The \textit{task-related cost} measures how much the system's current state and inputs deviate from their desired (reference) values. Minimizing this cost ensures the system stays close to the intended path during motion. The task-related cost function is formulated as:
\begin{equation}
L_{\text{task},k} = \frac{1}{2} \| W^x_k (\mathbf{x}_k - \mathbf{x}^{\text{ref}}_k) \|^2 + \frac{1}{2} \| W^u_k (\mathbf{u}_k - \mathbf{u}^{\text{ref}}_k) \|^2
\end{equation}
where $(\ast)^{\text{ref}}$ represents the reference (target) values for the state $\mathbf{x}_k$ and the control input $\mathbf{u}_k$. The reference trajectories were manually created by the method explained in Sec. \ref{sec:newsimulation} and describe the desired values for the CoM, base link, and end-effectors. The desired stiffness values are computed by solving the following least-squares optimization problem at each time step $k$:

\begin{equation}
\min \left\| \sum_l \lambda_{l,k}^2 \right\|^2 \quad \text{subject to} \quad \sum_l \lambda_{l,k}^2 (\mathbf{p}^{\text{ref}}_k - \mathbf{p}^{\text{ref}}_{l,k}) = \mathbf{g}
\end{equation}
This subproblem determines the stiffness distribution that supports the CoM against gravity. In addition, the desired values of the Centroidal Moment Pivot (CMP) offset and the moment of each end are typically set to zero unless non-zero values are specifically chosen to induce desired dynamic effects. The weighting matrices $W^x_k$ and $W^u_k$ are design parameters that control the importance given to state and input deviations in the cost function. Finally, when dealing with rotational variables represented by quaternions, the deviation between the actual and reference orientations is defined as :
\begin{equation}
\mathbf{q} - \mathbf{q}^{\text{ref}} := \omega(\mathbf{q}^{\text{ref}^{-1}} \mathbf{q})
\end{equation}
where $\omega(\cdot)$ maps a unit quaternion into an angle-axis vector.
\paragraph{Inequality Constraints}
In contact dynamics, physical conditions such as feasible positions, contact forces, and stiffness must be satisfied to ensure realistic motion. These are enforced through \textit{inequality constraints}, which maintain both physical plausibility and optimization feasibility. A detailed description of these constraints follows.\\
The position of each end link relative to the CoM and the base link is constrained using a box formulation:
\begin{equation}
\mathbf{p}_{\text{l,min}} \leq \mathbf{q}^{-1}(\mathbf{p}_l - \mathbf{p}) \leq \mathbf{p}_{\text{l,max}},
\end{equation}
Simple range constraints are imposed on the duration of each phase and the stiffness values:
\begin{equation}
    \tau_{\text{min}} \leq \tau \leq \tau_{\text{max}}, \tag{16}
\end{equation}
\begin{equation}
    0 \leq \lambda_l \leq \lambda_{\text{max}}, \quad \forall l. \tag{17}
\end{equation}
Next, for each end in contact, the contact wrench must satisfy non-slip and moment conditions.  
To prevent relative motion at the contact surface, sufficient friction must be maintained. This is achieved by requiring the contact force to lie within the friction cone. Specifically, the tangential force $f_t$ must satisfy 
\begin{equation}
    \lvert f_t \rvert \leq \mu f_n  \Longrightarrow \sqrt{f_{l,x}^2 + f_{l,y}^2} \leq \mu f_{l,z}
\end{equation}
where $\mu$ is the static friction coefficient. \\
Constraints on the moments at the contact point are expressed as:
\begin{align}
-c_{\text{max},x} f_{l,z} &\leq \eta_{l,x} \leq -c_{\text{min},x} f_{l,z} \\
c_{\text{min},y} f_{l,z} &\leq \eta_{l,y} \leq c_{\text{max},y} f_{l,z} \\
-\mu_z f_{l,z} &\leq \eta_{l,z} \leq \mu_z f_{l,z}
\end{align}
where $c_{\text{min}}$ and $c_{\text{max}}$ define the rectangular bounds of the center-of-pressure (CoP) region, and $\mu_z$ is the coefficient of friction torque.\\
All the inequality constraints can be compactly represented as:
\begin{equation}
g(\mathbf{x}_k, \mathbf{u}_k) \geq 0,
\end{equation}
where $g(\cdot)$ is a differentiable vector-valued function, evaluated componentwise. To handle these constraints during optimization, a log-barrier function is introduced:
\begin{equation}
L_{\text{limit}}(\mathbf{x}_k, \mathbf{u}_k) = \sum_{i=1}^{n_g} -\log \max(\epsilon, g_i(\mathbf{x}_k, \mathbf{u}_k)),
\end{equation}
where $n_g$ is the number of constraints, $g_i$ is the $i$-th constraint function, and $\epsilon$ is a small positive constant used to prevent numerical instability and avoid undefined values in the logarithmic function.
\paragraph{Contact Dependent Cost} 
To ensure consistent interaction between a robot’s end-effectors and the environment, a contact-dependent cost is introduced. It promotes complementarity between contact forces, velocities, and stiffness, encouraging physical consistency during contact and suppressing unnecessary interaction otherwise. This supports smooth transitions between contact and non-contact phases.\\
\newpage
The contact-dependent cost is defined as:
\begin{equation}
\label{eq:compl_cost}
\begin{aligned}
    J_{\text{compl},k} 
    &= w_{\text{compl}}^2 \sum_l \Bigg(
    \underbrace{ \sum_i \delta\left[\sigma_{l,k} = i\right] \left( \boldsymbol{\eta}_i^\top (\mathbf{p}_{l,k} - \mathbf{o}_i) \right)^2 }_{\text{contact distance constraint}} \\
    &\quad + \underbrace{ \delta\left[\sigma_{l,k} \neq \emptyset\right] \left( \| \mathbf{v}_{l,k} \|^2 + \| \boldsymbol{\omega}_{l,k} \|^2 \right) }_{\text{zero velocity constraint}} \\
    &\quad + \underbrace{ \delta\left[\sigma_{l,k} = \emptyset\right] \lambda_{l,k}^2 }_{\text{zero stiffness constraint}}
    \Bigg)
\end{aligned}
\end{equation}
Here $\sigma_{l,k}$ denotes the contact state of the $l$-th end at time step $k$, with $\sigma_{l,k} = i$ indicating contact with the $i$-th surface, and $\sigma_{l,k} = \emptyset$ indicating no contact. The operator $\delta[\ast]$ is an indicator function that returns $1$ if the condition inside the brackets is true, and $0$ otherwise.\\
\newline
Each term inside the cost function has a specific physical meaning. When the $l$-th end is in contact with surface $i$, the first term (top line of Eq. \eqref{eq:compl_cost}) penalizes the distance from the end’s position $\mathbf{p}_{l,k}$ to the surface origin $\mathbf{o}_i$ along the surface normal $\boldsymbol{\eta}_i$, enforcing proper alignment with the contact surface. The second term (middle line) becomes active whenever the end-effector is in contact with any surface and penalizes nonzero linear and angular velocities $\mathbf{v}_{l,k}$ and $\boldsymbol{\omega}_{l,k}$, thereby promoting static behavior at the contact point. Finally, when the end-effector is not in contact, the third term (bottom line) penalizes any nonzero stiffness $\lambda_{l,k}$, which prevents the generation of spurious contact forces during swing phases. \\
By properly tuning the weight parameter $w_{\text{compl}}$, these complementarity-related costs can be made negligible after optimization. A sufficiently large value of $w_{\text{compl}}$ ensures that the physical consistency conditions are respected without significantly penalizing the quality of the optimized trajectory.
\paragraph{Final Cost Function and Problem Formulation}
After defining the task-related, limit-related, and contact-dependent costs, the overall cost function is constructed by summing these individual terms over all time steps. It is defined as :
\begin{equation}
J[\boldsymbol{\sigma}] = \sum_k \left[ L_{\text{task},k} + L_{\text{limit},k} + L_{\text{compl},k}[\boldsymbol{\sigma}_k] \right]
\end{equation}
The planning problem can then be formulated as the following optimal control problem:
\begin{equation}
\begin{aligned}
& \text{find} \quad \mathbf{x}, \mathbf{u} \quad \text{that minimizes} \quad J[\boldsymbol{\sigma}](\mathbf{x},\mathbf{u}) \\
& \text{subject to} \quad \mathbf{x}_{k+1} = f(\mathbf{x}_k, \mathbf{u}_k)
\end{aligned}
\end{equation}
This formulation defines the optimal control problem to be solved for generating physically consistent and task-relevant trajectories.
\begin{comment}
This problem is formulated using a \textit{single shooting} approach, where the control inputs are directly optimized and the states are obtained by integrating the system dynamics forward in time. 
Due to the nonlinearities in the cost terms (especially those involving contact forces and constraints) and the system dynamics, the resulting optimization is generally a \textit{nonlinear programming problem} (NLP). However, in certain cases where the dynamics and cost functions are affine and the constraints are quadratic, the problem structure could be approximated as a \textit{quadratic programming} (QP) problem to enable faster computation.
Overall, the optimization simultaneously enforces task tracking, physical feasibility, and complementarity conditions related to contact interactions throughout the planned motion.
\end{comment}

\end{sloppypar}


\end{document}