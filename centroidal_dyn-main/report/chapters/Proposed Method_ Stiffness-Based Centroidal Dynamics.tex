\documentclass[main.tex]{subfiles}
\begin{document}

\section{Proposed Method}\label{sec:proposedmethods}
\subsection{Stiffness-Based Centroidal Dynamics}

We begin from the standard \emph{centroidal dynamics} equations, which relate the motion of the robot’s center of mass (CoM) to the total external wrench (force and moment) acting on the system:
\begin{subequations}\label{eq:centroidal‐dynamics}
\begin{align}
m\,\ddot p &= f \;-\; m\,g, \label{eq:centroidal‐dynamics:trans}\\
\dot L     &= \eta.      \label{eq:centroidal‐dynamics:rot}
\end{align}
\end{subequations}
Here:
\begin{itemize}
  \item $p\in\mathbb{R}^3$ is the position of the CoM, and $\ddot p$ its acceleration.
  \item $L\in\mathbb{R}^3$ is the total angular momentum about the CoM.
  \item $f\in\mathbb{R}^3$ and $\eta\in\mathbb{R}^3$ are, respectively, the translational and rotational components of the total external wrench.
  \item $m>0$ is the total mass, and $g\in\mathbb{R}^3$ is the gravity vector (e.g.\ $g=[0,0,-9.81]^\top$).
\end{itemize}

We assume all external wrenches are contact wrenches at $n_e$ end‐effectors (``ends'') of the robot.  Denote by $p_l\in\mathbb{R}^3$ the world‐frame position of the $l$-th end, and let
\[
f_l\in\mathbb{R}^3,\quad
\eta_l\in\mathbb{R}^3,\qquad l=1,\dots,n_e
\]
be the translational and rotational components of the contact wrench at that end.  Then the total wrench is
\begin{equation}\label{eq:sum‐wrench}
f    \;=\;\sum_{l=1}^{n_e} f_l,
\qquad
\eta \;=\;\sum_{l=1}^{n_e}\Bigl[(p_l - p)\times f_l \;+\;\eta_l\Bigr].
\end{equation}
The cross‐product in \eqref{eq:sum‐wrench} makes the system \emph{bilinear} in $(p,f_l)$, coupling CoM motion with contact forces.

\medskip
\paragraph{Spring-like parametrization of contact wrenches.}
Rather than holding $f_l,\eta_l$ constant, we introduce a \emph{stiffness} parameter $\lambda_l\ge0$ and a \emph{CMP‐offset} vector $r_l\in\mathbb{R}^3$ for each end, plus a pure‐moment direction $\hat\eta_l\in\mathbb{R}^3$.  Inspired by a spring model, we set:
\begin{equation}\label{eq:stiff‐param}
\boxed{
\displaystyle
f_l = m\,\lambda_l^2\Bigl(p \;-\;(p_l + r_l)\Bigr),
\qquad
\eta_l = m\,\lambda_l^2\,\hat\eta_l.
}
\end{equation}
Intuitively:
\begin{itemize}
  \item $\lambda_l^2$ scales like a contact stiffness: larger $\lambda_l$ → stronger repulsive force.
  \item $p_l + r_l$ is the \emph{virtual pivot} (similar to a Centroidal Moment Pivot, CMP).  The force pulls the CoM toward that point.
  \item $\hat\eta_l$ encodes any pure moment about the CoM, scaled consistently by $\lambda_l^2$.
\end{itemize}

\subsubsection*{Derivation of the stiffness‐based model}
By substituting \eqref{eq:sum‐wrench} and \eqref{eq:stiff‐param} into \eqref{eq:centroidal‐dynamics:trans} and neglecting $O(\epsilon^2)$ terms one obtains
\begin{align}
m\,\ddot p
&= \sum_{l=1}^{n_e} m\,\lambda_l^2\bigl(p - (p_l + r_l)\bigr) \;-\; m\,g
\notag\\
&\approx \sum_{l=1}^{n_e} m\,\lambda_l^2\bigl(p - (p_l + r_l)\bigr) \;-\; m\,g \;+\; m\,\epsilon^2 p
\notag\\
&= m\Bigl(\sum_{l}\lambda_l^2 + \epsilon^2\Bigr)p \;-\; m\Bigl(\sum_{l}\lambda_l^2(p_l + r_l) + g\Bigr)
\notag\\
&= m\,\bar\lambda^2\bigl(p - \bar p - \bar r\bigr),
\label{eq:deriv‐trans}
\end{align}
where
\[
\bar\lambda^2 = \sum_{l}\lambda_l^2 + \epsilon^2,
\quad
\bar p = \frac{\sum_{l}\lambda_l^2\,p_l + g}{\bar\lambda^2},
\quad
\bar r = \frac{\sum_{l}\lambda_l^2\,r_l}{\bar\lambda^2}.
\]

Similarly, substituting into \eqref{eq:centroidal‐dynamics:rot} gives
\begin{align}
\dot L
&= \sum_{l=1}^{n_e}\bigl[(p_l - p)\times m\,\lambda_l^2(p - p_l - r_l) + m\,\lambda_l^2\,\hat\eta_l\Bigr]
\notag\\
&= \sum_{l}\bigl[(p - p_l)\times m\,\lambda_l^2\,r_l\bigr]
   \;+\;\sum_{l}m\,\lambda_l^2\,\hat\eta_l
\notag\\
&= p \times m\,\bar\lambda^2\,\bar r
   \;+\;\sum_{l}m\,\lambda_l^2\bigl(\hat\eta_l - p_l\times r_l\bigr)
\notag\\
&\approx \bigl(m\,\ddot p + m\,\bar\lambda^2(\bar p+\bar r)\bigr)\times\bar r
   \;+\;\sum_{l}m\,\lambda_l^2\bigl(\hat\eta_l - p_l\times r_l\bigr)
\notag\\
&= m\bigl(\ddot p\times\bar r + \bar\eta\bigr),
\label{eq:deriv‐rot}
\end{align}
where
\[
\bar\eta
= \bar\lambda^2(\bar p\times\bar r)
  + \sum_{l}\lambda_l^2\bigl(\hat\eta_l - p_l\times r_l\bigr).
\]

\medskip
\paragraph{Closed‐form centroidal dynamics.}
Combining \eqref{eq:deriv‐trans} and \eqref{eq:deriv‐rot} yields the stiffness‐based centroidal equations:
\begin{subequations}\label{eq:stiff‐centroidal}
\begin{align}
\ddot p &= \bar\lambda^2\bigl(p - (\bar p + \bar r)\bigr),
\label{eq:stiff‐centroidal:trans}\\
\dot L &= m\bigl(\ddot p \times \bar r + \bar\eta\bigr).
\label{eq:stiff‐centroidal:rot}
\end{align}
\end{subequations}

\noindent
The aggregated parameters $\bar\lambda,\bar p,\bar r,\bar\eta$ recover the same expressions as before. 

\paragraph{Discussion and special cases.}
\begin{remark}[Exactness vs.\ flight phase]
If one ignores flight (i.e.\ always in contact, $\sum_l\lambda_l^2>0$), one may set $\epsilon=0$ and \eqref{eq:stiff‐centroidal} hold exactly.  Otherwise $\epsilon>0$ guarantees a well‐defined $\bar\lambda$ in airborne phases.
\end{remark}

\begin{remark}[Ballistic motion]
When all ends lose contact ($\lambda_l=0$ for all $l$), one finds
\[
\ddot p = \epsilon^2\,p - g \approx -\,g,
\quad
\dot L = 0,
\]
recovering the usual ballistic CoM motion and conservation of angular momentum.
\end{remark}

\begin{remark}[Relation to existing models]
Stiffness‐based (or force‐to‐point) parametrization has appeared before, but typically only at the \emph{total} wrench level.  Here we assign a separate $\lambda_l,r_l,\hat\eta_l$ to each end, which yields a unified multi‐contact description.  The classical CoP and (e)CMP emerge naturally as $\bar p$ and $\bar p+\bar r$, respectively.
\end{remark}

\subsection{Closed‐Form Solutions and Discrete‐Time Equations}

We subdivide the time horizon $[0,T]$ into $N$ consecutive intervals
\[
[t_k,\,t_{k+1}],\quad k=0,1,\dots,N-1,\qquad
t_{k+1}=t_k+\tau_k.
\]
We assume that \emph{contact states} (i.e.\ which ends are in contact) change only at the boundaries $t_k$.  Moreover, we apply a \emph{zero‐order hold} on the stiffness‐based parameters
\[
\{\lambda_{l}(t),\,r_{l}(t),\,\hat\eta_{l}(t)\}
\quad\mapsto\quad
\{\lambda_{l,k},\,r_{l,k},\,\hat\eta_{l,k}\}
\quad\text{for }t\in[t_k,t_{k+1}),
\]
meaning that each parameter is held constant over the interval.

\medskip
\paragraph{\ State at the beginning of interval $k$.}
Let
\[
p_k = p(t_k),\quad
v_k = \dot p(t_k),\quad
L_k = L(t_k),
\]
and compute the aggregated quantities
\[
\bar\lambda_k
=\sqrt{\sum_{l=1}^{n_e}\lambda_{l,k}^2+\epsilon^2},\quad
\bar p_k
=\frac{\sum_{l=1}^{n_e}\lambda_{l,k}^2\,p_{l,k}+g}{\bar\lambda_k^2},\quad
\bar r_k
=\frac{\sum_{l=1}^{n_e}\lambda_{l,k}^2\,r_{l,k}}{\bar\lambda_k^2},
\]
\[
\bar\eta_k
=\bar\lambda_k^2\,(\bar p_k\times\bar r_k)
\;+\;\sum_{l=1}^{n_e}\lambda_{l,k}^2\bigl(\hat\eta_{l,k}-p_{l,k}\times r_{l,k}\bigr).
\]

\paragraph{\ Analytical solution on $[t_k,t_{k+1}]$.}
With $\bar\lambda_k,\bar p_k,\bar r_k,\bar\eta_k$ constant, the CoM‐dynamics 
\[
\ddot p = \bar\lambda_k^2\bigl(p -(\bar p_k+\bar r_k)\bigr)
\]
is a linear second‐order ODE whose homogeneous+particular solution reads
\[
p(t)
= (\bar p_k+\bar r_k)
\;+\;C_k(\Delta t)\,\bigl(p_k -(\bar p_k+\bar r_k)\bigr)
\;+\;\frac{S_k(\Delta t)}{\bar\lambda_k}\,v_k,
\tag{7a}
\]
\[
v(t)
=\dot p(t)
=\bar\lambda_k\,S_k(\Delta t)\,\bigl(p_k -(\bar p_k+\bar r_k)\bigr)
\;+\;C_k(\Delta t)\,v_k,
\tag{7b}
\]
where $\Delta t = t - t_k$ and
\[
C_k(\Delta t) = \cosh\bigl(\bar\lambda_k\,\Delta t\bigr),\qquad
S_k(\Delta t) = \sinh\bigl(\bar\lambda_k\,\Delta t\bigr).
\]
Finally, substituting into the angular‐momentum equation
\[
\dot L = m\bigl(\ddot p\times \bar r_k + \bar\eta_k\bigr)
\]
and integrating from $t_k$ to $t$ gives
\[
L(t)
= L_k
\;+\;m\Bigl(\bigl(v(t)-v_k\bigr)\times\bar r_k
   \;+\;(t-t_k)\,\bar\eta_k\Bigr).
\tag{7c}
\]

\medskip
\begin{remark}[Zero‐Order Hold]
A zero‐order hold means we approximate time‐varying parameters by piecewise–constant values on each interval.  This yields closed‐form expressions above, at the cost of not capturing high‐frequency parameter variations.
\end{remark}

\bigskip
\subsection{Integration of Base‐Link Rotation}

The centroidal state $(p,\dot p,L)$ does not specify the \emph{orientation} $R(t)\in SO(3)$ or the \emph{base‐link} angular velocity $\omega(t)\in\Bbb R^3$.  In a multi‐body system one shows
\[
L \;=\; \underbrace{R\,I\,R^\top}_{\displaystyle I_{\rm sys}(R)}\,\omega
\;+\;R\,\hat L,
\tag{8}
\]
where
\begin{itemize}
  \item $I\in\Bbb R^{3\times3}$ is the composite inertia in the base‐link frame,
  \item $\hat L$ is the angular momentum about the base link due to internal motions.
\end{itemize}
If we fix reference values $I_{\rm ref},\,\hat L_{\rm ref}$ (e.g.\ from a nominal whole‐body motion), we solve for
\[
\omega(t)
= \;I_{\rm sys}(R)^{-1}\bigl(L - R\,\hat L_{\rm ref}\bigr)
\approx R\,I_{\rm ref}^{-1}\bigl(R^\top L - \hat L_{\rm ref}\bigr).
\tag{9}
\]

\paragraph{Discrete quaternion update.}
Let $q_k\in\mathbb{H}$ be the unit‐quaternion representing $R(t_k)$.
Over $[t_k,t_{k+1}]$ we subdivide into $n_{\rm div}$ equal steps
\[
t_k = t'_0 < \cdots < t'_i < \cdots < t'_{n_{\rm div}} = t_{k+1},\qquad
\tau_k' = \frac{\tau_k}{n_{\rm div}},\quad
t'_i = t_k + i\,\tau_k'.
\]
At each substep we assume $\omega$ nearly constant and update
\[
q_{k+1}
= \underbrace{q\bigl(\omega(t'_{n_{\rm div}})\,\tau_k'\bigr)}
_{\text{quat. for small rotation}}
\;\cdots\;
q\bigl(\omega(t'_1)\,\tau_k'\bigr)\;\cdot\;
q\bigl(\omega(t'_0)\,\tau_k'\bigr)\;\cdot\;q_k,
\tag{10}
\]
where $q(\theta)$ is the unit quaternion corresponding to the axis‐angle $\theta\in\Bbb R^3$.  The designer chooses $n_{\rm div}$ to balance integration accuracy against computational cost of gradient evaluation in trajectory optimization.


\end{document}