\documentclass[main.tex]{subfiles}

\begin{document}

\section{Related Works}\label{sec:relatedworks}
\begin{sloppypar}
Understanding Centroidal Dynamics is essential for controlling humanoid robots. Centroidal Dynamics \cite{orin2013centroidal} refers to the dynamics of a robot projected onto its Center of Mass (CoM),  which represents the average position of all its mass and serves as the point where the overall gravitational force can be considered to act. Although humanoid robots are complex systems with high-dimensional, nonlinear dynamics, the motion of their Center of Mass can often be described using simpler, more intuitive models. Focusing on the CoM allows the global dynamics of the robot to be captured without modeling each joint and link individually, significantly reducing system complexity. This approach provides an effective framework for planning, control, and stability, enabling the development of robots with robust, agile, and human-like behavior. Thanks to these advantages, many reduced order models have been developed by simplifying CD in different ways.\\
\paragraph{Simplified Centroidal Models} One of the foundational reduced-order models in humanoid locomotion is the Linear Inverted Pendulum Model (LIPM) \cite{kajita1991study}. This model treats the robot as a point mass and simplifies the equations of motion by assuming a constant height of the Center of Mass and neglecting angular momentum. These assumptions linearize the dynamics and make the model especially suitable for fast and efficient planning and control in legged robots. The model ensures stability by maintaining the Zero Moment Point (ZMP) within the support polygon defined by the feet. To improve upon LIPM and allow for more dynamic behaviors, the Variable-Height Linear Inverted Pendulum Model (VH-LIPM) \cite{caron2020biped} was introduced. This model integrates a linear feedback controller that aligns with the 3D Divergent Component of Motion (DCM) \cite{englsberger2015three} under feasible conditions and leverages vertical CoM variations when the ZMP nears the edge of the support region. Another widely adopted model is the Spring-Loaded Inverted Pendulum (SLIP), which assumes a compliant leg structure and is often used to replicate running dynamics \cite{full1999templates, holmes2006dynamics}. An extension of this, the Asymmetric SLIP (ASLIP) model \cite{poulakakis2009spring}, combines the flexibility of SLIP with the formal guarantees of Hybrid Zero Dynamics (HZD) \cite{westervelt2003hybrid} control theory to produce stable running motions. The ASLIP includes torso dynamics that are nontrivially coupled with leg motion, further enhancing its realism.
\paragraph{Integration of Trajectory Optimization} Reduced-order models like LIPM, VH-LIPM and SLIP are commonly embedded into trajectory optimization (TO) frameworks. TO involves computing optimal motion plans by minimizing a cost function subject to dynamic and physical constraints. In these setups, reduced models serve as simplified system representations within these optimization problems. Approaches to TO for CoM trajectories vary. Some use LIPM as the state equation \cite{englsberger2017smooth, kamioka2018simultaneous}, defining desired ZMP as a cost \cite{tedrake2015closed} and enforcing stability through ZMP constraints \cite{kajita2003biped, murooka2022centroidal}. When both CoM movement and base link rotation are considered, centroidal dynamics is used either as a constraint or state equation \cite{kuindersma2016optimization, rossini2023real}, with stability enforced via support criteria like the ZMP region or the Centroidal Wrench Cone (CWC) \cite{ding2021nonlinear, dai2016planning}.
\paragraph{Limitations and Solutions} Despite their flexibility, numerical trajectory optimization techniques often require small time steps for accurate integration, leading to a large number of decision variables and increased computational cost. Additionally, these methods usually guarantee feasibility only at discrete time points, leaving feasibility in between unverified. Some attempts have been made to mitigate these issues by linearizing centroidal dynamics—typically by neglecting rotational effects and fixing the CoM height \cite{audren2014model}. While these simplifications can be effective in conventional walking scenarios, they fall short when tackling more dynamic tasks like multi-contact planning or acrobatic maneuvers. To address the limitations of discretization-heavy methods, researchers have explored closed-form solutions of centroidal dynamics. A common method involves treating contact wrenches as piecewise constant (zero-order hold), which enables larger integration intervals. However, this often induces undesirable angular momentum fluctuations unless extremely short time steps are used. In \cite{tazaki2022fast}, the multi-contact (mc-) LIPM is proposed, expressing contact forces as functions of stiffness and the displacement between contact points and the CoM. This method allows for larger integration steps—potentially spanning entire contact phases—without significant angular momentum disturbances. \\
\\
To overcome these drawbacks, a novel stiffness-based method is introduced \cite{tazaki2024trajectory}. This method derives closed-form solutions for centroidal dynamics by parameterizing contact wrenches through stiffness models. These analytical solutions are then integrated into trajectory optimization, enabling the generation of longer, dynamically feasible trajectories with significantly fewer decision variables.
\end{sloppypar}
\end{document}